\documentclass[12pt, a4paper, onecolumn, titlepage]{article}
% Encoding
%--------------------------------------
\usepackage[utf8]{inputenc}
\usepackage[T1]{fontenc}
%--------------------------------------
 
% German-specific commands
%--------------------------------------
\usepackage[ngerman]{babel}
%--------------------------------------
 
% Hyphenation rules
%--------------------------------------
\usepackage{hyphenat}
\hyphenation{Mathe-matik wieder-gewinnen}
%--------------------------------------

% Included packages
%--------------------------------------
\usepackage{times}
\usepackage{csquotes}
\usepackage[onehalfspacing]{setspace}
\usepackage[
backend=biber,
style=alphabetic
]{biblatex}
\usepackage{tocloft}


\usepackage{subfiles}
%--------------------------------------


\addbibresource{Quellen.bib}

\title{Technologie - Wird Künzliche Intelligenz Zu Unserem Ende Führen?}
\author{Mihály Papp \\
  \multicolumn{1}{p{.7\textwidth}}{\centering\emph{Friedrich Schiller Gymnasium und Schülerwohnheim, Klasse 12/5}}}

\date{Prüfungsjahren 2020-2021}

\begin{document}

% Deckblatt
\maketitle

% Inhaltsverzeichnis
\renewcommand{\cftpartleader}{\cftdotfill{\cftdotsep}}
\tableofcontents
\newpage

% Begründung der Themenwahl
\subfile{Sub-dokumenten/Begruendung der Themenwahl.tex}
\newpage

% Quellenverzeichnis
\section{Benutzte Quellen}
\printbibliography[title={" "}, heading=subbibliography]
\newpage

\part{Inhaltsangaben}
% Textwiedergabe 1: Künzliche Intelligenz: Ende der Monotonie
\subfile{Sub-dokumenten/Kuenzliche Intelligenz: Ende der Monotonie.tex}
\newpage

\end{document}