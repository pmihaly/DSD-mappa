% !TEX root = ../Mappe.tex
\documentclass[../Mappe.tex]{subfiles}

\begin{document}
\section[Künzliche Intelligenz: Ende der Monotonie]{Künzliche Intelligenz: Ende der Monotonie \cite{zeitki}}

Das letzte drei Jahrhundert der Geschichte handelt sich um die Automatisierung.
Neulich wird diese informatische Erscheinung mit künzliche Intelligenz bewehrt.
Der Artikel von Die Zeit, geschrieben von Zukunftsexperten Daniel Dettling und Matthias Horx,
beseitigt die Missverständnissen,
und führt die Wirkungen dieser Erfindung aus.
Ich möchte nun der Inhalt wiedergeben.

Zuerst wird über die Warnungen der Welt-berümter Wissenschaftler Stephen Hawking und der Unternehmer Elon Musk berichtet.
Sie beide halten künzliche Intelligenz zu gefährlich um er zu benutzen.

Ein Teil die deutsche Aufregung protestiert gerade gegen die Technologie: 
Über je mehr Tätigkeit verfügen sich unsere elektronische Geräte, desto weniger unsere kognitive Laste, meint der gelehrte Mittelschicht.
Das heißt in Sprache der Panik: \emph{'Wir werdem immer dümmer!'}, \emph{'Die Maschinen nehmen unsere Jobs weg!'}
und andere rechtmässige Bemerkungen, als auch falsche Auslegungen.

Weiterhin der Artikel beschreibt uns der doppeltes Interpretation des Begriffes 'künzliche Intelligenz'.
Einerseits bezieht diese Ausdruck auf der informatische Problemenlösungsmethode, wobei der Denkweise der menschliche Gehirn wird imitiert\footnote{Das wird auch als 'maschinelles Lernen' genannt}.
Andererseits eine künzliche Bewusstsein mit Gefühlen und Kreativität.
Obwohl allein denkende und fühlende Geräte noch in Fantasieland leben, die Technologie, was der erste Definition beschreibt, wird heute täglich benutzt. 

In der nächste Absatz wird über die Fähigkeiten der künstliche Intelligenz gesprochen.
Dank diese Technologie, können wir Automatisierung mit Prognostizierungsfähigkeiten ausrüsten.
Ein Automatisierungssystem mit maschinelles Lernen wird alles mögliche für die 'bester Vorfall' tun.
Der Wort 'beste' wird von Musterbeispiele bei der Algorithmus für sich definieren.

Immer breitere Arbeitsrollenvielfalt und immer kreativitätenbedürfige Berufen werden mit Computer erledigt.
Das heißt, wir Menschen werden immer überflüssiger für der Arbeitsmarkt.

Doch die Algorithmen können uns nicht von Lehrstellenmarkt auslöschen, denn sie haben keines soziale Gefühlen und gesellschaftlicher Position.
Der Zukunftserfolg wird in Vereinigung dieser künzliche Intelligenz und die menschliche Verlhältnisse sich befinden.
\end{document}