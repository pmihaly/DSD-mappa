% !TEX root = ../Mappe.tex
\documentclass[../Mappe.tex]{subfiles}

\begin{document}
\section{Künzliche Intelligenz: Ende der Monotonie}

Für die dritte Jahrhundert gleichzeitig kämpfen und verwirklichen wir die Automatisierung.
Einerseits machen wir uns eine kollektive Bequemheit und verursachen wir unsere eigene Wirtschaftshölle.

In die Artikel von Die Zeit, geschrieben von Daniel Dettling und Matthias Horx, wird über mit künzliche Intelligenz
entwickelte Automatisierung diskutiert.
Ich möchte nun der Inhalt wiedergeben.

Zuerst wird über die Warnungen von der Welt-berümte Wissenschaftler Stephen Hawking und der multi-Unternehmer Elon Musk berichtet.
Sie beide halten künzliche Intelligenz zu gefährlich um die Früchte dieser Technologie zu ernten.

Ein Teil die deutsche Aufregung protestiert gerade gegen die Technologie: 
Über je mehr Tätigkeit verfügen sich unsere elektronische Geräte, desto weniger unsere kognitive Laste, meint der gelehrte Mittelschicht.
Das heißt in Sprache der Panik: \emph{'Wir werdem immer dümmer!'}, \emph{'Die Maschinen nehmen unsere Jobs weg!'}
und andere rechtmässige Bemerkungen, als auch falsche Auslegunge.

Weiterhin beschreibt uns der Artikel der doppeltes Interpretation des Begriffes 'künzliche Intelligenz'.
Einerseits heißt diese Ausdruck die informatische Problemenlösungsmethode, wobei der Denkweise der menschliche Gehirn imitiert wird\footnote{Das wird auch als 'maschinelles Lernen' genannt}.
Andererseits eine künzliche Bewusstsein mit Gefühlen und Kreativität.
Obwohl allein denkende und fühlende Geräte noch in Fantasieland existieren, die Technologie, was der erste Definition beschreibt, wird heute täglich benutzt. 

In der nächste Artikel wird über die fähigkeiten der künstliche Intelligenz gesprochen.
Dank diese Technologie, kann Automatisierung mit Prognostizierungsfähigkeiten ausgerüstet werden.
Ein Automationssystem mit maschinelles Lernen wird alles mögliche für die 'beste' Zukunft tun.
Das Interpretation der Wort 'beste' wird von Musterbeispiele bei der Algorithmus formuliert.

Immer breitere Arbeitsrollenvielfalt und immer kreativitätbedürfige Berufen werden mit Computer erledigt.
Das heißt, wir werden immer überflüssiger für der Arbeitsmarkt.

Doch die Algorithmen können uns nicht von Lehrstellenmarkt nicht auslöschen:
Sie haben keinen Gefühlen und gesellschaftliche Position.
Der Zukunftserfolg wird in Vereinigung der künzliche Intelligenz und menschliche Verlhältnisse sich befinden.
\end{document}